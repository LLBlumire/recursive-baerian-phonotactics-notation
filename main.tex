\documentclass{scrartcl}

\usepackage[a4paper, margin=1.3in]{geometry}
\usepackage{amsmath}
\usepackage{float}
\usepackage{libertine}
\usepackage{relsize}
\usepackage[UKenglish]{isodate}

\date{1st August 2017}

\newcommand{\nulls}[0]{\symbol{"2205}}

\title{Recursive Baerian Phonotactics Notation}
\author{L. L. Blumire}

\begin{document}
\pagenumbering{gobble}

\maketitle

\vspace{-3em}
\begin{center}
Edited by Sascha M. Baer
\end{center}
\vspace{1em}


Recursive Baerian\footnote{/ˈbɛɹiən/} Phonotactics Notation (A.K.A Baerian Notation, RBPN) was invented by Sascha M. Baer on the 1st May 2017 for describing the phonotactics of the constructed language `Qahfó'. It allows for rigorous but simple definitions, and is easily extensible.

An example of it can be seen in this definition of the constructed language `Xekela':

\[
\#
    \bigg[_{\hspace{0.2em}\omega}
        \Big[_{\hspace{0.1em}\sigma}
            \substack{\displaystyle \text{C  }\vspace{2pt}\\\displaystyle \text{PF  }} 
            \substack{\displaystyle \text{V  }\vspace{2pt}\\\displaystyle \hspace{-2.5pt}\text{D}}
            \substack{\displaystyle \text{C  }\vspace{2pt}\\\displaystyle \hspace{-2.5pt}\text{\nulls}}
            \substack{\displaystyle \vspace{1pt}\\\mathlarger{\sigma}\vspace{3pt}\\\displaystyle \text{\nulls}\vspace{2pt}}
        \Big]
    \bigg]
\# 
\]

Written inline\footnote{Inline notation designed by L. L. Blumire (author)} as $\#\left[_\omega \left[_\sigma \substack{\text{C}\\\text{P}\cdot\text{F}} \cdot \substack{\text{V}\\\text{D}} \cdot \substack{\text{C}\\\text{\nulls}} \cdot \substack{\sigma\\\text{\nulls}} \right] \right]\#$ where C is consonants, V is vowels, P is plosives, F is fricatives, and D is a specific set of diphthongs present in the language.

This might be written under more traditional notation as C(C)V(V)(C), or under a slightly more useful notation\footnote{Regular expressions} as ((C|PF)(V|D)?(C)?)+.

The notation has a number of components, the beginning definition $\#[$ which marks the start of a word, and $]\#$ which marks the end of a word. Bracket pairs mark blocks, every block is given an assigned variable. By convention the word level block is assigned $\omega$, foot level block is assigned $\varphi$, and syllable level blocks are assigned $\sigma$, such that $[_\alpha \ldots ]$ defines a block $\alpha$.

Inside blocks, characters with predefined definitions are given, for example with C for consonant, and V for vowel. $\#[_\omega \text{C} \cdot \text{V} ]\#$ would define simple phonotactic system where all words are one syllable long and all are of the structure CV.

To create optional consonants, or choices between possible subsets, choices are stacked. For example $\#\left[_\omega \substack{\text{C}\\\text{\nulls}} \cdot \text{V} \right]\#$ which would define (C)?V --- however still only allowing one syllable per word.

To allow for multiple syllables per word, recursion can be used. $\#\left[_\omega \left[_\sigma \substack{\text{C}\\\text{\nulls}} \cdot \text{V} \cdot \substack{\sigma\\\text{\nulls}} \right] \right]\#$ describes ((C)?V)+ such that multiple syllables can occur in a word; the usage of $\sigma$ on the right hand side refers to a substitution of the block defined as $\sigma$ in the centre of the definition. This defines phonotactics for an optional initial consonant, followed by a vowel, repeated any number of times in a word.

For reference, the following is the original use case of the notation, describing `Qahfó'. $\#\left[_\omega \substack{\text{V}\\\text{\nulls}} \left[_\varphi \left[_{\sigma_1} \substack{\text{C}\\\text{P}\cdot\text{L}} \cdot \text{V} \cdot \substack{\text{ʔ}\\\text{ː}} \right] \left[_{\sigma_2} \substack{\text{C}\\\text{P}\cdot\text{L}} \cdot V \right] \substack{\varphi\\\text{\nulls}} \right] \right]\#$; or in a prettier form:

\[
\#
\Bigg[_\omega
\substack{\displaystyle \text{V}\vspace{2pt}\\\displaystyle \text{∅}}
\bigg[_\varphi
	\Big[_{\sigma_1}
		\substack{\displaystyle \text{C  }\vspace{2pt}\\\displaystyle \text{PL  }} 
		\text{V  } 
		\substack{\displaystyle \text{ʔ}\vspace{3pt}\\\displaystyle \text{ː}}
	\Big]
	\Big[_{\sigma_2}
		\substack{\displaystyle \text{C  }\vspace{2pt}\\\displaystyle \text{PL  }} 
		\text{V  } 
	\Big]
	\substack{\displaystyle \varphi\vspace{2pt}\\\displaystyle \text{∅}}
\bigg]
\Bigg]\# 
\]

\end{document}
